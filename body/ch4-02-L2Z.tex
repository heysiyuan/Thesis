

Consider the case where the penalty function $ f_j $ for each group $ j $ is linearly decreasing before its desired allocation amount $ m_j $ and 0 afterwards. Namely, we have:
\begin{align*}
    f_j(x) = \begin{cases}
                 f_j^L(x) = -\beta_j x + \beta_j m_j, x\in [0, m_j], \\
                 f_j^R(x) = 0, x\in [m_j, \infty),
             \end{cases}
\end{align*}


The offline optimization problem can then be expressed as follows:
\begin{subequations}\label{equation-linear2zero-soft-gfq-primal}
    \begin{align}
        \max_{0 \leq x_t \leq r_t}\quad
         & \sum_{t \in [T]} v_tx_t -
        \sum_{j \in [K]}
        \beta_j
        \Bigl[\,m_j - \sum_{t=1}^{T} x_t \cdot\mathbf 1_{\{j_t = j\}}\Bigr]^{+} \\[6pt]
        \text{s.t.}\quad
         & \sum_{t=1}^{T} x_t \leq 1
    \end{align}
\end{subequations}

\paragraph{Remark} We may take an alternative view of the optimization problem by considering it as a subsidy of $ \beta_j $ for the first $ m_j $ units of allocation to group $ j $. In this case, the optimization problem can be reformulated as follows,
\begin{subequations}\label{equation-linear2zero-subsidy}
    \begin{align}
        \max_{0 \leq x_t \leq r_t}\quad
         & \sum_{t \in [T]} (v_t + \beta_{j_t}) x_t -
        \sum_{j \in [K]}
        \beta_j
        \Bigl[\sum_{t \in [T]} x_t \cdot \mathbf{1}_{\{j_t = j\}} - m_j\Bigr]^{+} \\[6pt]
        \text{s.t.}\quad
         & \sum_{t=1}^{T} x_t \leq 1
    \end{align}
\end{subequations}.

\begin{assumption}[High Subsidy Regime]\label{assumption-linear-to-zero-high-subsidy}
    The penalty parameters are sufficiently large such that the minimum subsidy across all groups, when combined with the lowest possible valuation, exceeds the maximum valuation:
    \begin{align}
        1 + \min_{j \in [K]} \beta_j \geq \theta.
    \end{align}
    Equivalently, this can be expressed as $\theta - \min_{j \in [K]} \beta_j \leq 1$.
\end{assumption}


This assumption ensures that even arrivals from the group with the smallest penalty parameter $\beta_j$ receive sufficient subsidy to make their effective valuation $(1 + \beta_j)$ competitive with the highest possible natural valuation $\theta$.

\subsection{Sufficient Conditions}

\subsection{Necessary Conditions}

The hard instance for linear-to-zero penalties is similar to the zero-to-linear case. We again consider a \textit{Semi-Adaptive} adversary that chooses the hard instance based on the initial reservation for each group.



\begin{definition}[Hard Instance Family for Soft GFQ with Linear-to-Zero Penalty]
    Let $\Pi^K$ denote the set of all permutations of $[K]$. For any permutation $\pi \in \Pi^K$ and sufficiently small $\epsilon > 0$, we define the hard instance $I^{\text{L}\to\text{Z}}_\pi$ as a sequence of $K+1$ batches of arrivals constructed as follows:
    \begin{enumerate}
        \item \textbf{Initial batch:} One arrival from each group $j \in [K]$ with value $v = 1$, all arriving simultaneously.
        \item \textbf{Subsequent batches:} For each group $j \in [K]$ in order $\pi_1, \pi_2, \ldots, \pi_K$, a continuum of arrivals with values continuously increasing from $ 1 + \epsilon $ to $ \theta $.
    \end{enumerate}

    Formally, the instance is represented as:
    \begin{align*}
        I^{\text{L}\to\text{Z}}_\pi =  \Biggl\{
        \underbrace{(1,\pi_1), (1, \pi_2), \dots (1, \pi_K)}_{\text{Initial batch}},
         & \underbrace{(1+\epsilon, \pi_1), (1 + 2\epsilon, \pi_1), \dots, (\theta, \pi_1)}_{\text{Arrivals for group $\pi_1$}}, \dots, \\
         & \underbrace{(1+\epsilon, \pi_K), (1 + 2\epsilon, \pi_K), \dots, (\theta_K, \pi_K)}_{\text{Arrivals for group $\pi_K$}}
        \Biggr\},
    \end{align*}
    where each element $(v,j)$ represents an arrival from group $j$ with valuation $v$.
\end{definition}

The key observation is that the initial batch remains identical across all instances $I^{\text{L}\to\text{Z}}_\pi$ for any permutation $\pi \in \Pi^K$. Consequently, any optimal online algorithm must make allocation decisions that are robust against the adversarial choice of arrival order in subsequent batches.

\begin{theorem}[Necessary Conditions for Linear-to-Zero Penalty]\label{theorem-linear2zero-soft-gfq-necessary}
    Let $\alpha > 1$ be a competitive ratio. If there exists an $\alpha$-competitive online algorithm for all instances $I^{\text{L}\to\text{Z}}_\pi$ with $\pi \in \Pi^K$, then there must exist utilization functions $\psi^\pi_j: [1, \theta] \to [0, b_j]$ for each group $j \in [K]$ and permutation $\pi \in \Pi^K$, where the budgets for each group $\{b_j\}_{j \in [K]}$ satisfies $\sum_{j\in[K]}b_j \leq 1$, such that the following system of inequalities holds:

    \begin{subequations}\label{equation-linear2zero-necessary-conditions}
        \begin{gather}
            \sum_{j\in [K]} \Psi^\pi_j(1) \geq \frac{1}{\alpha}\left((1-M+m_{\pi_1})v_{\pi} + \sum_{j=2}^K m_{\pi_j}\right), \label{equation-linear2zero-necessary-initial-condition}\\[0.5em]
            \Psi^\pi_{\pi_1}(v_{\pi_1})+\sum_{j=2}^K \Psi^\pi_{\pi_j}(1) \geq \frac{1}{\alpha} \left(\sum_{j=2}^K m_{\pi_j} +(1-M+m_{\pi_1})v_{\pi_1}\right), \label{equation-linear2zero-necessary-first-group-condition}\\
            \forall v_{\pi_1} \in [v_\pi, \theta], \nonumber\\[0.5em]
            \sum_{j=1}^{i-1} \Psi^\pi_{\pi_j}(\theta) + \Psi^\pi_{\pi_i}(v_{\pi_i}) +\sum_{j=i+1}^K \Psi^\pi_{\pi_j}(1) \geq \frac{1}{\alpha} \left((1-M+\sum_{j=1}^{i-1}m_{\pi_j})\theta + m_{\pi_i}\cdot v_{\pi_i}+\sum_{j=i+1}^K m_{\pi_j}\right), \label{equation-linear2zero-necessary-general-condition}\\
            \forall i\in[2,K], \forall v_{\pi_i} \in [1+\epsilon, \theta], \nonumber\\[0.5em]
            \sum_{j\in [K]}\psi^\pi_j (\theta) \leq 1, \label{equation-linear2zero-necessary-boundary-condition}
        \end{gather}
    \end{subequations}
    for all permutations $\pi \in \Pi^K$.

    \noindent where:
    \begin{itemize}
        \item $\Psi^\pi_j(v) := \psi^\pi_j(1)+\int_1^v u \, d\psi^\pi_j(u) - f_j(\psi^\pi_j(v))$ represents the net utility function for group $j$ under permutation $\pi$,
        \item $M := \sum_{j\in[K]}m_j$ denotes the total desired allocation across all groups,
        \item $f_j(\cdot)$ is the linear-to-zero penalty function for group $j$.
    \end{itemize}
\end{theorem}

\begin{proof}
    The proof follows by considering the performance of any $\alpha$-competitive algorithm against the adversarial instances.

    \paragraph{Initial Condition \eqref{equation-linear2zero-necessary-initial-condition}} After the first batch of arrivals, the online algorithm's performance can be captured by the left-hand side of \eqref{equation-linear2zero-necessary-initial-condition}. With the second batch of arrivals for group $\pi_1$, there must exist a critical value $v_\pi \in (1, \theta]$ such that:
    \begin{itemize}
        \item Up until $ v_\pi $, i.e., $ \forall v < v_\pi $, the left-hand side of \eqref{equation-linear2zero-necessary-initial-condition} strictly exceeds the right-hand side, indicating that the current allocation is sufficient to guarantee $ \alpha $ competitiveness.
        \item At $v = v_\pi$, equality holds in \eqref{equation-linear2zero-necessary-initial-condition}, marking the point where the algorithm must begin allocating to maintain competitiveness.
        \item After $ v_\pi $, the algorithm must allocate more resources to prevent the right-hand side from exceeding the left-hand side, which would then violate the $\alpha$-competitiveness requirement.
    \end{itemize}

    Meanwhile, with the initial batch and the second batch for group $ \pi_1 $ with values up to $ v_\pi $, the offline optimal is to allocate $ m_{\pi_j} $ amount of resources to each group $ \pi_j \in [K] $, and allocate the remaining portion $ 1 - M $ to class $ \pi_1 $ with value $ v_{\pi} $, leading to the optimal value as described on the right-hand side of \eqref{equation-linear2zero-necessary-initial-condition}.

    \paragraph{Incremental Condition (I) \eqref{equation-linear2zero-necessary-first-group-condition}} In the second batch of arrival for group $ \pi_1 $, as the value of arrivals increase beyond  $ v_\pi $, the online algorithm must maintain $ \alpha $-competitiveness against the offline optimal, which can be expressed as followed for any $ v_{\pi_1} \in [v_\pi, \theta] $:
    \begin{align*}
        \OPT(I^{\text{L}\to\text{Z}}_\pi) \leq (1-M+m_{\pi_1})v_{\pi_1} + \sum_{j=2}^K m_{\pi_j}.
    \end{align*}
    On the other hand, the performance of the online algorithm is
    \begin{align*}
        \ALG(I^{\text{L}\to\text{Z}}_\pi) \geq \Psi^\pi_{\pi_1}(v_{\pi_1})+\sum_{j=2}^K \Psi^\pi_{\pi_j}(1),
    \end{align*}
    which implies second necessary condition.

    \paragraph{General Incremental Condition \eqref{equation-linear2zero-necessary-general-condition}}
    We now analyze the general case where batches for groups $\pi_1, \ldots, \pi_{i-1}$ have already arrived, we are currently processing arrivals from group $\pi_i$ with maximum observed value $v_{\pi_i} \in [1+\epsilon, \theta]$, and batches for groups $\pi_{i+1}, \ldots, \pi_K$ are yet to arrive.

    The offline optimal strategy in this scenario is to construct an allocation that maximizes total welfare by:
    \begin{itemize}
        \item Allocating $m_{\pi_j}$ units to each completed group $\pi_j$ for $j \in \{1, \ldots, i-1\}$, selecting arrivals with the maximum value $\theta$ from their respective batches.
        \item Allocating $m_{\pi_i}$ units to the current group $\pi_i$, selecting the arrival with value $v_{\pi_i}$.
        \item Allocating $m_{\pi_j}$ units for each group $\pi_j$ for $j \in \{i+1, \ldots, K\}$, using the arrivals with value 1 from the initial batch.
        \item Allocating the remaining capacity $(1-M+\sum_{j=1}^{i-1}m_{\pi_j})$ to any arrival with value $ \theta $ in previous batches, regardless of their group.
    \end{itemize}
    This allocation strategy leads to the following upper bound on the offline optimal performance for any $ v_{\pi_i} \in [1+\epsilon, \theta] $:
    \begin{align*}
        \OPT(I^{\text{L}\to\text{Z}}_\pi) \leq (1-M+\sum_{j=1}^{i-1}m_{\pi_j})\theta + m_{\pi_i} v_{\pi_i}+\sum_{j=i+1}^K m_{\pi_j}.
    \end{align*}
    Meanwhile, the performance of the online algorithm is
    \begin{align*}
        \ALG(I^{\text{L}\to\text{Z}}_\pi) \geq \sum_{j=1}^{i-1} \Psi^\pi_{\pi_j}(\theta) + \Psi^\pi_{\pi_i}(v_{\pi_i}) +\sum_{j=i+1}^K \Psi^\pi_{\pi_j}(1),
    \end{align*}
    which implies the third necessary condition.

    \paragraph{Boundary condition \eqref{equation-linear2zero-necessary-boundary-condition}}
    The final condition ensures that the total allocation across all groups does not exceed the available capacity. Since $\psi^\pi_j(\theta)$ represents the maximum allocation that can be made to group $j$ under permutation $\pi$ when the highest-value arrivals are observed, the constraint $\sum_{j\in [K]}\psi^\pi_j (\theta) \leq 1$ directly enforces the overall capacity constraint.
\end{proof}

From Theorem \ref{theorem-linear2zero-soft-gfq-necessary}, we can observe that the initial reservations $ \psi^\pi_j(1) $ for each group $ j $ and the arrival order $ \pi $ play a key role. To obtain the lower bound on the competitive ratio $ \alpha $, we need to solve an optimization problem that satisfies the necessary conditions in \eqref{equation-linear2zero-necessary-conditions} for all permutations $ \pi \in \Pi^K $. Below, we provide a case study for the case of two groups with equal penalty parameters $ \beta_1 = \beta_2 = \beta $.

\begin{corollary}
    [Lower Bound for Two Groups with Linear-to-Zero Penalties]\label{theorem-linear2zero-soft-gfq-lower-bound-two-groups}
    For $ K = 2 $ groups with linear-to-zero penalty functions and equal penalty parameters $ \beta_1 = \beta_2 = \beta $, under the family of hard instances $ I^{\text{L}\to\text{Z}}_\pi $, no online algorithm can achieve a competitive ratio better than $ \alpha^* $, where $ \alpha^* $ is the optimal value of the following optimization problem:
    \begin{subequations}\label{equation-linear2zero-two-groups-optimization}
        \begin{align}
            \min \quad        & \alpha                                                                                                                                                        \label{equation-linear2zero-two-groups-objective}   \\[0.5em]
            \text{s.t.} \quad & \frac{1-M+m_{\pi_1}}{\alpha}\ln\left(\frac{v^*_{\pi_1} +\beta}{v_{\pi}+\beta}\right) + \psi_{\pi_1}(1) = m_{\pi_1},                                           \label{equation-linear2zero-two-groups-constraint1} \\[0.3em]
                              & \frac{m_{\pi_2}}{\alpha}\ln\left(\frac{v^*_{\pi_2} +\beta}{1+\beta}\right) + \psi_{\pi_2}(1) = m_{\pi_2},                                                     \label{equation-linear2zero-two-groups-constraint2} \\[0.3em]
                              & 1 \geq m_1 + m_2 + \frac{1-M+m_{\pi_1}}{\alpha}\ln\left(\frac{\theta}{v^*_{\pi_1}}\right)+\frac{m_{\pi_2}}{\alpha}\ln\left(\frac{\theta}{v^*_{\pi_2}}\right), \label{equation-linear2zero-two-groups-constraint3} \\[0.3em]
                              & \sum_{j\in[2]}\left[\psi_{\pi_j}(1)-\beta(m_{\pi_j} - \psi_{\pi_j}(1))^+\right] \geq \frac{1}{\alpha}\left((1-M+m_{\pi_1})v_\pi + m_{\pi_2}\right), \label{equation-linear2zero-two-groups-constraint4}
        \end{align}
    \end{subequations}  for all $\pi\in\Pi^2$.
\end{corollary}

\begin{proof}
    % First, we can rewrite the system of inequalities in Theorem \ref{theorem-linear2zero-soft-gfq-necessary} for the case of two groups ($K=2$) with equal penalty parameters ($\beta_1 = \beta_2 = \beta$).

    % \paragraph{Reduction to Binding Relations} The necessary conditions in Theorem \ref{theorem-linear2zero-soft-gfq-necessary} are a system of inequalities. Let $v_\pi \in (1,\theta]$ denote the first value at which allocation to $\pi_1$ becomes strictly positive. On any interval where $\psi^\pi_{\pi_i}$ is (right-)differentiable and strictly increasing, the corresponding inequality must hold with equality; otherwise a local decrease of the allocation would preserve feasibility and strictly improve $\alpha$. Thus equalities hold precisely on active intervals and at $v_\pi$ for the initial condition. We therefore retain the original inequality directions:
    % \begin{subequations}\label{equation-linear2zero-two-groups-necessary}
    %     \begin{align}
    %         \sum_{j\in [2]} \Psi^\pi_j(1)                          & \ge \frac{1}{\alpha}\left((1-M+m_{\pi_1})v_{\pi} + m_{\pi_2}\right)                                                             &  & \text{(tight at } v_{\pi}\text{)}, \label{equation-linear2zero-two-groups-initial}                      \\
    %         \Psi^\pi_{\pi_1}(v_{\pi_1})+\Psi^\pi_{\pi_2}(1)        & \ge \frac{1}{\alpha} \left(m_{\pi_2} +(1-M+m_{\pi_1})v_{\pi_1}\right), \quad \forall v_{\pi_1} \in [v_\pi, \theta]              &  & \text{(tight when } \psi^\pi_{\pi_1} \text{ increases)}, \label{equation-linear2zero-two-groups-first}  \\
    %         \Psi^\pi_{\pi_1}(\theta) + \Psi^\pi_{\pi_2}(v_{\pi_2}) & \ge \frac{1}{\alpha} \left((1-M+m_{\pi_1})\theta + m_{\pi_2} v_{\pi_2}\right), \quad \forall v_{\pi_2} \in [1+\epsilon, \theta] &  & \text{(tight when } \psi^\pi_{\pi_2} \text{ increases)}, \label{equation-linear2zero-two-groups-second} \\
    %         \sum_{j\in [2]}\psi^\pi_j (\theta)                     & \le 1                                                                                                                           &  & \text{(may be slack)}. \label{equation-linear2zero-two-groups-capacity}
    %     \end{align}
    % \end{subequations}

    % For $i\in\{1,2\}$, where $\psi^\pi_{\pi_i}$ is absolutely continuous,
    % \[
    %     \Psi^\pi_{\pi_i}(v) = \psi^\pi_{\pi_i}(1) + \int_1^v u\,d\psi^\pi_{\pi_i}(u) - f_{\pi_i}(\psi^\pi_{\pi_i}(v))
    % \]
    % yields, by differentiation,
    % \begin{align*}
    %     \frac{d}{dv}\Psi^\pi_{\pi_i}(v) & = v\frac{d\psi^\pi_{\pi_i}}{dv} - f'_{\pi_i}(\psi^\pi_{\pi_i}(v))\frac{d\psi^\pi_{\pi_i}}{dv}           \\
    %                                     & = \frac{d\psi^\pi_{\pi_i}}{dv}\left[v - f'_{\pi_i}(\psi^\pi_{\pi_i}(v))\right]                          \\
    %                                     & = \begin{cases}
    %                                             (v + \beta)\frac{d\psi^\pi_{\pi_i}}{dv}, & \text{if } \psi^\pi_{\pi_i}(v) < m_{\pi_i},    \\
    %                                             v\frac{d\psi^\pi_{\pi_i}}{dv},           & \text{if } \psi^\pi_{\pi_i}(v) \geq m_{\pi_i}.
    %                                         \end{cases}
    % \end{align*}

    % On any active interval for group $\pi_1$, differentiating \eqref{equation-linear2zero-two-groups-first} gives
    % \begin{align*}
    %     \frac{d\Psi^\pi_{\pi_1}}{dv_{\pi_1}} = \frac{1-M+m_{\pi_1}}{\alpha}.
    % \end{align*}
    % Let $v^*_{\pi_1}$ be the critical value where $\psi^\pi_{\pi_1}(v^*_{\pi_1}) = m_{\pi_1}$. For $v_{\pi_1} \in [v_\pi, v^*_{\pi_1}]$, we have:
    % \begin{align*}
    %     (v_{\pi_1} + \beta)\frac{d\psi^\pi_{\pi_1}}{dv_{\pi_1}} = \frac{1-M+m_{\pi_1}}{\alpha} \quad \Rightarrow \quad \frac{d\psi^\pi_{\pi_1}}{dv_{\pi_1}} = \frac{1-M+m_{\pi_1}}{\alpha(v_{\pi_1}+\beta)}.
    % \end{align*}
    % Integrating from $v_\pi$ to $v^*_{\pi_1}$ and using $\psi^\pi_{\pi_1}(v^*_{\pi_1}) = m_{\pi_1}$:
    % \begin{align*}
    %     m_{\pi_1} - \psi^\pi_{\pi_1}(v_\pi) = \frac{1-M+m_{\pi_1}}{\alpha}\ln\left(\frac{v^*_{\pi_1} + \beta}{v_\pi + \beta}\right).
    % \end{align*}
    % Since $\psi^\pi_{\pi_1}(v_\pi) = \psi^\pi_{\pi_1}(1)$ (no allocation before $v_\pi$), this gives \eqref{equation-linear2zero-two-groups-constraint1}.

    % An identical argument applied to \eqref{equation-linear2zero-two-groups-second} yields \eqref{equation-linear2zero-two-groups-constraint2} for $\pi_2$ on its active interval(s).

    % For $v > v^*_{\pi_i}$, we obtain
    % \begin{align*}
    %     v\frac{d\psi^\pi_{\pi_i}}{dv} = \frac{1-M+m_{\pi_1}}{\alpha} \quad \Rightarrow \quad \psi^\pi_{\pi_i}(v) = m_{\pi_i} + \frac{1-M+m_{\pi_1}}{\alpha}\ln\left(\frac{v}{v^*_{\pi_i}}\right).
    % \end{align*}

    % Substituting these expressions: (i) evaluating at $v=v^*_{\pi_i}$ gives \eqref{equation-linear2zero-two-groups-constraint1}--\eqref{equation-linear2zero-two-groups-constraint2}; (ii) inserting $\psi^\pi_{\pi_i}(\theta) = m_{\pi_i} + \frac{1-M+m_{\pi_1}}{\alpha}\ln(\theta/v^*_{\pi_i})$ in the capacity inequality yields \eqref{equation-linear2zero-two-groups-constraint3}; and (iii) using $\Psi^\pi_j(1)=\psi^\pi_j(1)-\beta(m_{\pi_j}-\psi^\pi_j(1))^+$ in the initial inequality gives \eqref{equation-linear2zero-two-groups-constraint4}.

    % Since the optimization problem must hold for both permutations $\pi \in \Pi^2$, the result follows.

\end{proof}


\subsection{Connections to Hard GFQ Constraints}