This thesis studied Online Fractional Knapsack with both packing costs and group quotas, where resources are allocated to arriving agents in an online manner. Using the competitive analysis framework, we investigated the interplay between maximizing allocation efficiency and enforcing group quota constraints.

First, we examined the setting in which hard quota constraints are imposed on each group of agents alongside a global packing cost that limits the total allocation. We developed online algorithms that achieve tight competitive ratios and provided matching lower bounds, illustrating how these two requirements interact in algorithm design.

Moving beyond hard constraints, we explored surrogate penalty functions to implement soft quota constraints. These functions impose costs for both under-allocation and over-allocation to each group of agents, and can be viewed as taxes and subsidies in the allocation process. We designed competitive algorithms for several implementations of these soft quota constraints: subsidies for under-allocation, taxes for over-allocation, and a combination of both. Additionally, we constructed hard instance families to establish lower bounds on the competitive ratios achievable under these settings and provide intuition for these limitations.

In conclusion, this thesis contributes to our understanding of the combination of efficiency maximization and group fairness in online allocation problems through the competitive analysis framework, specifically in the context of Online Fractional Knapsack with packing costs and group quotas. Future research could explore more sophisticated soft quota implementations, such as non-linear penalty functions that progressively penalize deviations from desired allocation levels. The trade-off between allocation efficiency and deviations from quota targets could also be studied more systematically, potentially enabling automated design of penalty parameters to optimize this trade-off on a Pareto frontier. Lastly, the design of truthful mechanisms with monetary transfers in strategic settings, where agents may misreport their values or group affiliations, remains an open and intriguing direction. Integrating incentive compatibility with efficiency and fairness objectives may yield new insights and challenges for online allocation.