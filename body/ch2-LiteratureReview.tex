In this chapter, we review the existing literature related to online allocation problems, and examine how our work fits into the broader context of research in this area.

\section{Online Allocation}

We begin by surveying the general landscape of online allocation problems, highlighting key models and assumptions that have been extensively studied in the literature. Next, we consider the body of work that takes into account cost functions in the decision-making problem and their practice relevance. Lastly, we explore various notions of fairness measures, both at the individual and group levels, that have been proposed to ensure fair outcome of the allocation process.

\subsection{Adversarial and Stochastic Settings in Online Allocation}

The study of online allocation problems can be broadly categorized into adversarial and stochastic settings, each with its own applications and challenges.

In adversarial settings, the input sequence is chosen by an adversary, making it essential for online algorithms to perform well against worst-case scenarios. Classic works in this area include the online knapsack problem \cite{chakrabarty2008online}, AdWords problems \cite{mehta2007adwords, mehta2013online}, and one-way trading problems \cite{el2001optimal}. The core challenge lies in making irrevocable decisions without knowledge of future inputs, requiring a careful balance between capturing immediate rewards and preserving resources for future opportunities based on worst-case considerations. This approach is particularly popular in the computer science community, where robustness against the unpredictable future is often prioritized, with applications in resource allocation in cloud-computing, online advertising, and network routing.

In contrast, stochastic settings assume that the input sequence is drawn from a known or unknown probability distribution. This allows online algorithms to future leverage the statistical properties of the input in its decision-making. Notable works in this stream include the prophet inequality framework initiated by Krengel and Sucheston \cite{krengel1978semiamarts}, with close ties to the optimal stopping theory \cite{samuel1984comparison}, online matching with unknown distributions \cite{karande2011online}, and online stochastic knapsack problems \cite{jiang2022tight}. The goal this regime is to design algorithms that maximize the expected performance, often by learning and adapting to the underlying distribution over time. The stochastic approach is particularly relevant in operations research and economics, where decision-making under uncertainty is a central theme, with applications in inventory management, revenue management, and online marketplaces.

\subsection{Online Allocation with Costs}

Traditionally, online allocation problems have focused on maximizing the social welfare or revenue from the allocation process, assuming that the resources are readily available at no cost. However, in many real-word scenarios, there are costs associated with producing or procuring resources, which must be taken into account in the allocation decisions. For example, in cloud computing, there are meaningful power and cooling costs associated with sever usage \cite{lin2011threshold}; in network bandwidth allocation, congestion and latency costs increase with higher link utilization \cite{christodoulou2005price}. Cost functions have been introduced to many online allocation models to capture these practical considerations, including the online allocation with costs \cite{tan2020mechanism,tan2025threshold}, and online matching with concave returns \cite{devanur2012online}.

Incorporating cost functions into online allocation problems introduces another layer of complexity, as algorithms must now also balance the trade-off between maximizing rewards and minimizing costs. Excessive premature resource allocation may lead to prohibitively high costs that overwhelm the allocation process later on. Therefore, online algorithms must consider not only the immediate rewards but also its impact on future costs, making the decision-making process more intricate.

\subsection{Individual Fairness in Online Allocation}

Originated in the works of Steinhaus \cite{steinhaus1949division} and Varian \cite{varian1973equity}, fair division addresses the fundamental task of dividing limited resources among multiple agents in a fair manner. Many fairness notions have been introduced to characterize the desired fairness properties, including proportionality \cite{steinhaus1949division}, envy-freeness \cite{gamonpuzzle}, maximin share \cite{hill1987partitioning}, and competitive equilibrium from equal incomes \cite{budish2011combinatorial}. Traditionally, fair division problems have been studied in the offline setting, where all agents and their preferences are known a priori. Even so, the design of fair allocation schemes often prove to be challenging, particularly when resources are indivisible, which have led to various impossibility results and different relaxations to aforementioned fairness notions, such as envy-freeness up to one good (EF1) \cite {lipton2004approximately}, envy-freeness up to any good (EFx) \cite{caragiannis2019unreasonable}, and approximation schemes such as $ \alpha $-approximation of maximin share \cite{amanatidis2017approximation}.

With the increasing prevalence of online platforms that match resources to users in real time, a growing body of work introduces fairness notions into online allocation. A fundamental tension often arises between fairness and efficiency, since enforcing fair outcomes may reduce achievable social welfare or revenue. Nevertheless, maximizing the Nash social welfare has been shown to yield fair and efficient allocations of indivisible goods: outcomes satisfy EF1 and Pareto optimality and offer strong approximation guarantees for the maximin share \cite{caragiannis2019unreasonable}. For divisible goods, online algorithms have been developed that achieve envy-freeness and approximately optimal social welfare for two agents \cite{gkatzelis2021fair}. As in the offline setting, designing fair online allocation schemes remains elusive, frequently leading to relaxations or approximations of fairness notions or optimization objectives, or to model modifications such as allowing reassignment of items \cite{he2019achieving}.

\subsection{Group Fairness in Online Allocation}

In many applications, there is a natural partition of agents into different groups, such as demographic, geographic, or socioeconomic groups. To consider the fairness of the allocation outcomes at the group level, there are two main approaches: the public goods and public decision making setting where all agents within a group derive utility from the resources allocated to the group or the resulting public decision \cite{segal2019fair, conitzer2017fair, fain2018fair}, and the private goods setting where resources allocated to a group are further distributed among the agents within the group \cite{conitzer2019group, aleksandrov2018group}. In both settings, the designers often come from a fair division perspective, where the goal is to develop or approximate group fairness notions.

In contrast, there have been few works that study group fairness while maximizing the efficiency of the online allocation process. At its core, it's akin to multi-objective optimization, where the goal is to maximize both the overall social welfare and some desired fairness measures. More recently, \cite{10.1145/3695411.3695414} incorporates group fairness into the revenue-maximizing online conversion problems by guaranteeing that each group receives a predetermined quantity of resources. Similarly, \cite{10.1145/3726854.3727299} considered different notions of group fairness and highlights the fundamental tension between individual welfare and group fairness.

\section{Analysis Frameworks and Techniques}
In this section, we present the main analytical frameworks and techniques used to study online allocation problems. These tools underpin both the design and the analysis of online algorithms, and form the theoretical basis for the work in this thesis.

\subsection{Competitive Analysis in Online Algorithms}

Competitive analysis is a standard framework for evaluating the performance of online algorithms by comparing them to an optimal offline benchmark that has complete knowledge of the entire input sequence in advance. The performance measure in this framework is the \emph{competitive ratio}, which quantifies the worst-case performance loss of the online algorithm due to the lack of future information.

For maximization problems, an online algorithm is said to be $\alpha$-competitive (for some $\alpha \ge 1$) if for every input sequence $I$,
\begin{equation*}
    \mathbb{E}[\text{ALG}(I)] \ge \alpha \cdot \text{OPT}(I),
\end{equation*}
where $\mathbb{E}[\text{ALG}(I)]$ denotes the expected value achieved by the online algorithm on $I$, and $\text{OPT}(I)$ is the value of an optimal offline algorithm on the same input. Here, a smaller $\alpha$ (i.e., a larger fraction of $\text{OPT}$) corresponds to better performance.

For minimization problems, an online algorithm is $\alpha$-competitive if for every input sequence $I$,
\begin{equation*}
    \mathbb{E}[\text{ALG}(I)] \le \alpha \cdot \text{OPT}(I),
\end{equation*}
with the terms defined analogously. Similarly, a smaller competitive ratio indicates a smaller worst-case performance loss relative to the offline optimum.

In both cases, the expectation is taken over the interval randomness of the algorithm in case of randomized online algorithms, while the input sequence could be chosen adversarially to induce the maximum performance gap.

The competitive analysis framework has been extensively applied to a wide range of online allocation problems, most related to this thesis including the online resource allocation with supply cost \cite{tan2020mechanism}, online allocation with multi-class arrivals \cite{10.1145/3726854.3727299}, and online matching with concave returns \cite{devanur2012online}. The competitive ratio provides consistent performance measures that allows for meaningful comparisons between the hardness of different online allocation problems and the effectiveness of various algorithmic strategies.


\subsection{Online Primal-Dual Framework}
In the design and analysis of online algorithms, the online primal-dual framework \cite{buchbinder2009design} has emerged as a powerful technique. Not only is it a general and unifying approach that can be applied to a wide range of online problems, but it also provides meaningful insights and interpretations of the algorithm's behavior through the lens of linear programming duality. Originated in the seminal work of Goemans and Williamson \cite{goemans1995general} for approximation algorithms, the online primal-dual framework has evolved into a systematic approach for online optimization problems, such as online ad-auctions \cite{buchbinder2007online}, online covering and packing \cite{buchbinder2006improved, buchbinder2009online}, online matching \cite{kalyanasundaram2000optimal, ma2020algorithms,devanur2012online}, and a variety of online optimization problems \cite{bartal2003incentive, blum2011welfare, huang2019welfare, tan2020mechanism, tan2025threshold, huang2025long}.

The online primal-dual framework maintains a corresponding pair of primal and dual variables that are updated as new input arrives, whilst ensuring that the primal solution remains feasible and the dual solution maintains optimality conditions. The dual variables often service as prices or penalities that guide the online algorithm's decisions, which are dynamically adjusted based on the current state of the system. Therefore, the design of competitive online algorithms usually comes down to designing a set of update rules for the dual variables and therefore their trajectory. In this thesis, we leverage this approach to design and analyze the performance of our algorithms, with the benefit of interpreting the dual values as prices to draw insights into the system dynamics.

