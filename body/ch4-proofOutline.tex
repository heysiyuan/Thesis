In this section, we provide an outline of the proofs for the sufficient and necessary conditions presented in section \ref{section-main-results-soft-quota}. We will defer the full proofs to the appendix.
\subsection{Proof Outline for Sufficient Conditions}

The proof of the sufficient conditions in section \ref{section-main-results-soft-quota} follows the primal-dual framework. There are three main steps in the proof: dual feasibility, incremental inequality, and primal feasibility. We will outline each step below.

\emph{Dual feasibility} requires showing that the dual variables are feasible at each time step, which can be easily verified in the update rules of the Algorithms.

\emph{Incremental inequalities} involve analyzing the incremental changes in the primal and dual objectives at each time step, due to the update of the dual variables and the allocation decision made by the algorithm. We want to show that at each time step $ t $, the change in the primal objective $ \Delta P_t $ is at least $ 1/\alpha $ times the change in the dual objective $ \Delta D_t $, i.e., $ \Delta P_t \geq \Delta D_t/\alpha $. We consider different cases based on the value of the arriving agent $ v_t $ relative to the dual variables. In each case, we derive expressions for $ \Delta P_t $ and $ \Delta D_t $, and verify the incremental inequality.

\emph{Primal feasibility} is the final step, where we need to show that the allocation decisions produced by the algorithm are feasible, i.e., the total allocation does not exceed the capacity $ B $. This is the most involved part of the proof, where we use induction on time $ t $ and the idea of \emph{future feasibility}. We define a max-potential allocation function that captures the total allocation up to time $ t $ plus the maximum potential future allocation allowed by the current state of the dual variables. By showing that this max-potential allocation function remains within the capacity $ B $ for all time steps, we can conclude that the total allocation is feasible.


\subsection{Proof Outline for Necessary Conditions}

The proof of necessary conditions in section \ref{section-main-results-soft-quota} follows directly by analyzing the performance of any online algorithm's utilization against the family of hard instances $ \{I^{\SOFTQUOTA}\}_{\forall j \in [K]} $ as defined in Definition \ref{definition-hard-instance-soft-quota}. The key is that these hard instances are indistinguishable to the online algorithm in the first stage of arrivals, the algorithm needs to provide $ \alpha $-competitiveness guarantee against all possible $ j \in [K] $ in the second stage of arrivals.


