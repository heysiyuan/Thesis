
Online allocation problems involve the irrevocable distribution of scarce resources to agents arriving sequentially, without prior knowledge of future arrivals. A classical objective in this domain is the maximization of total agent utility, often referred to as social welfare. The central challenge stems from uncertainty regarding future arrivals, which forces decision-makers to balance allocating resources to current agents against preserving them for potentially higher-value future requests. However, many real-world applications introduce two additional critical components. First, resources are rarely free; they typically incur production or procurement costs that must be integrated into allocation decisions. Second, quota requirements are frequently imposed to ensure desirable outcomes, particularly when agents are partitioned into distinct groups with fairness concerns. In such contexts, maximizing aggregate utility is insufficient, as fairness across different groups must also be prioritized.

In this thesis, we develop frameworks and algorithms that explicitly address these components in Online Fractional Knapsack problems. First, we introduce fairness-by-quantity measures within a framework of online knapsack with production costs and hard-quota constraints, meaning that a minimum quota must be strictly met and a cost is incurred for each additional unit of items being packed. Second, extending this analysis beyond a single group of items, we further study packing schemes that incentivize outcomes satisfying group quota requirements via quota-regulation based costs. Under this scheme, violating the group quota is permissible, but incurs an increased regulatory cost (e.g., in the form of tax). Collectively, this work provides a novel perspective at the intersection of efficiency maximization, cost management, and group fairness in online allocation problems.