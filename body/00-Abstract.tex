
Online allocation problems address the challenge of making irrevocable decisions to allocate scarce resources to a sequence of agents arriving over time, without knowledge of future arrivals. A classical objective in these problems is to maximize the total utility of agents receiving resources, often referred to as the social welfare. The key difficulty then arises from the uncertainty of future arrivals: the decision maker must carefully balance the trade-off between allocating resources to current agents and preserving resources for potentially higher-value future agents. Many real-world applications, however, introduce two more critical components to the online allocation problem. First, resources are often not readily available for free to be allocated, but instead incur production or procurement costs that must be considered in the allocation decisions. Second, quota requirements are often imposed to achieve desirable outcomes, particularly when agents are naturally partitioned into distinct groups. In such settings, it is no longer sufficient to simply maximize the aggregate utility. The fairness in the allocation outcomes across different groups must also be taken into account.

In this thesis, we develop frameworks and algorithms that explicitly capture these two components in online allocation problems. First, we introduce fairness-by-quantity measures within the online allocation with production costs framework, where a minimum quota must be allocated and a cost is incurred for each unit of resources produced. Second, extending beyond a single group, we use cost functions as surrogates to study how to design allocation schemes that incentivize the allocation outcome towards satisfying group quota requirements. Together, our work provides a novel perspective at the intersection of efficiency maximization, cost management, and group fairness in online allocation problems.