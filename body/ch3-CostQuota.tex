\section{Problem Statement}

We consider a seller with a total resource budget $B$ and a production cost function $f:[0,B]\to\mathbb{R}_{+}$. The seller offers the divisible resource to a sequence of buyers indexed by $t\in\{1,\dots,T\}$, with the total number of buyers $T$ unknown. Each buyer $t$ is characterized by a per-unit value $v_t$ and an allocation rate limit $r_t$, and arrives sequentially over time.

When buyer $t$ arrives, the seller must irrevocably choose an allocation decision $x_t\in[0,r_t]$ without any information about future buyers. Let $\{x_t\}_{t=1}^T$ denote the resulting allocation, which must satisfy two key constraints.

First, the total allocation cannot exceed the resource budget:
\[
    \sum_{t\in[T]} x_t \leq B.
\]
Second, there is an allocation quota $M$ that must be met:
\[
    \sum_{t\in[T]} x_t \geq M.
\]

Given an allocation $\{x_t\}_{t=1}^T$, the social welfare is defined as the total value derived from the buyers minus the production cost incurred by the seller:
\[
    \sum_{t\in[T]} v_t x_t - f\left(\sum_{t\in[T]} x_t\right).
\]

\subsection{Problem Assumptions}

We make the following assumptions regarding the problem setting:

\begin{assumption}[Bounded Valuation]\label{assumption-bounded-value}
    For each buyer $t \in [T]$, the per-unit value $v_t$ lies within a known bounded interval $[1, \theta]$, where $\theta \geq 1$ is a constant.
\end{assumption}

\begin{assumption}[Quota Feasibility]\label{assumption-quota-feasibility}
    There exist at least $n$ arrivals, where
    \[
        n = \arg\min_{k} \left\{ \sum_{t=1}^{k} r_t \,\middle|\, \sum_{t=1}^{k} r_t \geq M \right\}
    \]
    denotes the smallest integer such that the quota requirement is feasible.
\end{assumption}

\begin{assumption}[Quota Profitability]\label{assumption-quota-profitability}
    We assume $f'(M) \leq 1$ (or equivalently, for all $v \in [1, \theta]$, $B(v) \geq M$, or $\underline{B} \geq M$) to ensure that it is always profitable for the seller to satisfy the quota requirement.
\end{assumption}

\subsection{Problem Formulation}

Formally, a problem setting can be defined by the following two components:

\begin{itemize}
    \item A \emph{setup} $S = \bigl(B,\,M,\,f,\,\theta\bigr)$ that includes information known a priori to the seller, where:
          \begin{itemize}
              \item $B > 0$ is the total resource budget,
              \item $M \in (0,B]$ is the allocation quota,
              \item $f : [0,B] \to \mathbb{R}_{+}$ is the production cost function,
              \item $\theta \ge 1$ is the upper bound on buyers' per-unit valuations (Assumption~\ref{assumption-bounded-value}).
          \end{itemize}

    \item An \emph{instance} $I = \bigl\{(v_t, r_t)\bigr\}_{t=1}^{T}$ denoting the sequence of arriving buyers, where:
          \begin{itemize}
              \item Each buyer $t$ is defined by a per-unit value $v_t \in [1,\theta]$ and an allocation rate limit $r_t \ge 0$,
              \item The total number of buyers $T$ is unknown a priori.
          \end{itemize}
\end{itemize}

\subsection{Offline Optimal Solution}

Let $\Omega$ denote the set of all possible arrival instances satisfying Assumptions~\ref{assumption-bounded-value}, \ref{assumption-quota-feasibility}, and \ref{assumption-quota-profitability}. For any instance $I \in \Omega$, the offline optimal performance $\OPT(I)$ can be obtained by solving the following optimization problem:
\begin{subequations}\label{equation-quota-primal}
    \begin{align}
        \max_{x_t\in[0, r_t]} \quad & \sum_{t\in [T]} v_t x_t - f\left(\sum_{t\in[T]} x_t\right) \\
        \text{subject to} \quad     & \sum_{t\in[T]} x_t \leq B,                                 \\
                                    & \sum_{t\in[T]} x_t \geq M.
    \end{align}
\end{subequations}

Our design objective is to develop online algorithms that are competitive against $\OPT(I)$ for any instance $I \in \Omega$ while satisfying the budget and quota constraints.

\section{Preliminaries}
Given a setup $ S = \bigl(B,\,M,\,f,\,\theta\bigr)$, we define the following notations that will be used throughout the chapter.




\section{Sufficient Conditions}

\section{Necessary Conditions}
